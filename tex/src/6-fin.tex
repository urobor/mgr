\chapter{Podsumowanie i~wnioski}

Praca ta miała na celu zidentyfikowanie optymalnej metody
odczytu numeru z~nadjeżdżającego autobusu oraz jej implementację
na urządzeniu mobilnym z~systemem Android. Zaimplementowano
rozwiązanie oparte na 12 kaskadowych detektorach oraz pętli 
dopasowań wzorców. Jednak nie stwierdzono czy metoda której użyto
jest optymalna. Jedyne co można o~niej powiedzieć to to, że
skuteczność jest na dostatecznym poziomie wyjściowym do dalszych 
usprawnień. Dzięki modułowemu podejściu można stopniowo ją
zwiększać poprzez poprawianie i~dopracowywanie poszczególnych
elementów. 

\section{Wyniki}

Ostatecznym wynikiem jest utworzenie szkieletu aplikacji
składającej się z~trzech elementów ułożonych kaskadowo, które
mogą być uruchomione równolegle celem zwiększenia szybkości wykonywania:

\begin{enumerate}
    \item Detektor frontu.
    \item Detektor numeru we froncie.
    \item Czytnik numeru.
\end{enumerate}

Detektor frontów podczas pisania tego tekstu prezentował skuteczność
na więcej niż zadowalającym poziomie.
Detektor numeru musiał zostać wspomożony ograniczeniem obszaru wyszukiwania
do lewego górnego rogu odnalezionego frontu autobusu.
Jest to element którego istnienie, oraz próba przetestowania
wykazała jak istotna jest liczność zbiorów uczących i~testowych.
Wykorzystując uczenie maszynowe w~celu lokalizacji i~odczytywania
tekstu z~tak zwanej sceny naturalnej wymaganym jest posiadania
zbiorów danych liczących dziesiątki, a~nawet setki tysięcy 
przykładów.

Detektory poszczególnych cyfr wprowadzone zostały w~celu ograniczenia
złożoności obliczeniowej podczas dopasowywania wzorców. Ich skuteczność
jest daleka od oczekiwanej lecz problemy podczas trenowania detektorów
o~większej ilości próbek (negatywnych i~pozytywnych)
utrudniają znacząco przygotowywanie nowych instancji.

\section{Problemy i~utrudnienia}

Największym utrudnieniem okazało się błędne podejście do zagadnienia. 
O~ile nie do uniknięcia było wykonanie doświadczeń i~eksperymentów,
które zakończyły się jedynie poszerzeniem posiadanej wiedzy.
To błędnym było wykonywanie ogromnej ilości prób i~pomiarów na
wczesnym etapie poszukiwań odpowiedniego rozwiązania. 
Wynikiem takiego podejścia były testy wykonane przy zbyt małej 
wiedzy na temat zagadnienia i~architektury końcowej programu. 

Rzetelne testy wydajności i~dokładności należało wykonać na końcu
gdy wiadomo było już jakie algorytmy, narzędzia i~biblioteki zostaną
zastosowane w~rozwiązaniu docelowym. 

W~celu rozwiązania tego problemu wykonano drugie podejście,
głównie skupiając się na przygotowaniu zestawów danych testowych
i~narzędzi do automatycznej weryfikacji skuteczności.
Podczas drugiej próby, przez wykorzystanie dwóch zestawów
testowych, zidentyfikowano problem zbyt małej ilości próbek w~nich
zwartych. Przeprowadzono udaną optymalizację,
i~weryfikację drugiej wersji detektora
numerów na jednym zestawie testowym (3000 próbek). Po czym okazało się, że na
innym zestawie (1903 próbki) pierwsza wersja wykazuje się większą 
skutecznością. 

\section{Dalsze prace}

Pierwszym krokiem w~celu przygotowania aplikacji na poziomie 
skuteczności, niezawodności i~użyteczności pozwalającym na 
wydanie komercyjne jest przygotowanie środowiska testowego.
Posiadanie kilkudziesięciu tysięcy oznaczonych i~opisanych
próbek wydaje się być sensowym punktem wyjścia.
Jest to niezbędne w~celu rzetelnej weryfikacji wprowadzanych 
usprawnień.

Możliwe są zmiany w~implementacji niosące za sobą większą funkcjonalność.
Może to być na przykład wykrywanie innych typów frontów niż 
tych z~wyświetlaczem. Cel można osiągnąć poprzez zastosowanie 
dodatkowych, specjalnie w~tym celu przygotowanych detektorów. Można też 
podjąć próbę implementacji detektorów tablic bocznych a także tyłów 
autobusów, gdzie oba te fragmenty zawierają numer linii.

Usprawnieniem i~pewnego rodzaju ułatwieniem odczuty byłoby zawężenia
możliwych do odczytania numerów do zbioru z~konkretnego przystanku. 
Współrzędne przystanków wraz z~listą numerów byłyby umieszczane 
w~centralnej bazie. Aplikacja mogła by pobierać aktualne 
współrzędne z~urządzenia GPS zainstalowanego w~telefonie i~na ich podstawie
ograniczać możliwy do odczytania zbiór do pobranej listy numerów.

Poprawę działania algorytmu można również osiągnąć wykorzystując 
lepszy telefon, zawierający lepszej jakości kamerę oraz
większą moc obliczeniową. Na rynku dostępne są już urządzenia
wyposażone w~procesory ośmiordzeniowe z~kamerami umożliwiającymi 
nagrywanie filmów w~rozdzielczości 4K.

Niewielką zmianą możliwą do zaimplementowania jest też wprowadzenie
tłumaczeń odczytywanych komunikatów na język użytkownika. 
Jest to zmiana raczej kosmetyczna jednak znacząca jeżeli chodzi
o~aplikacje komercyjne.

Ostatecznie istnieje szereg możliwości wzbogacania i~dodawania
funkcjonalności aplikacji w~oparciu o~użyte technologie.
Można na przykład dopisać moduł lokalizacji przystanków
na podstawie znaków - zadanie o~wiele łatwiejsze. Wymagało by to
wyszkolenia kolejnego detektora. W~tym przypadku, detektora znaków
występujących przy przystankach. Aplikacja przy wykorzystaniu 
modułu GPS była by w~stanie doprowadzić osobę do przystanku.

Można dodać obsługę rozpoznawania mowy, w~celu zadania punktu docelowego.
Przy tak zdefiniowanych kryteriach aplikacja używając komunikatów
głosowych prowadziła by użytkownika do najbliższego przystanku. 
Asystowała by też przy poprawnym ustawieniu telefonu w~celu odczytania
numeru. Podczas podróży możliwe było by aktualizowanie czasu oraz
odległości pozostałej do osiągnięcia celu. O~czym użytkownik informowany
był by przy pomocy osobnych komunikatów głosowych.

Jak widać możliwości rozbudowywania proponowanego narzędzia są niemal
nieskończone. Jedynym kryterium jakie powinno obowiązywać przy
doborze funkcji do zakodowania jest ich przydatność dla użytkowników 
końcowych. To oni tak na prawdę powinni być największymi beneficjentami 
potencjalnego systemu i~bez konsultacji z~przynajmniej jedną osobą
niewidomą nie ma wręcz sensu wdrażać rozwiązania, które mogą 
okazać się zupełnie nieprzydatne.

Największą
wartością dodaną tego opracowania jest stwierdzenie możliwości odczytania
numeru nadjeżdżającego autobusu przez urządzenie mobilne.
Osiągnięcie poziomu niezawodności pozwalającego osobie niewidomej 
na swobodne poruszanie się i~korzystanie z~transportu miejskiego jest
realne lecz wymaga ogromnego nakładu pracy. Pytaniem nad którym 
trzeba się poważnie zastanowić jest sens istnienia takiego programu.
Technologie typu OCR są zawodne, a~w~środowisku nie laboratoryjnym
zależą od wielu czynników zewnętrznych: oświetlenia, zniekształceń,
odblasków, przeszkód i~wielu innych. Pierwsze systemy 
OCR wykorzystywano m.in. do odczytywania danych z~kart kredytowych.
Jednak zaraz po pojawieniu się pasków magnetycznych zrezygnowano
z~technologii OCR na rzecz bardziej niezawodnych 
czytników magnetycznych. Z~drugiej strony zaletą OCR jest zerowy
koszt dodatkowej infrastruktury oraz elastyczność rozwiązania.
Dodatkowo prace naukowców na zbiorze SVHN pokazują, że 
dziedzina przetwarzania obrazów w~połączeniu z~sieciami neuronowymi
może mieć przed sobą jeszcze wiele znaczących osiągnięć.
