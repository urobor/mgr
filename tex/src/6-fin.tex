\chapter{Podsumowanie i~wnioski}

Praca ta miała na celu zidentyfikowanie optymalnej metody
odczytu numeru z~nadjeżdżającego autobusu oraz jej implementację
na urządzeniu mobilnym z~systemem Android. Zaimplementowano
rozwiązanie oparte na 12 kaskadowych detektorach oraz pętli 
dopasowań wzorców. Jednak nie stwierdzono czy metoda której użyto
jest optymalna. Jedyne co można o~niej powiedzieć to to, że
skuteczność jest na dostatecznym poziomie wyjściowym do dalszych 
usprawnień. Dzięki modułowemu podejściu można stopniowo ją
zwiększać poprzez poprawianie i~dopracowywanie poszczególnych
elementów. 

\section{Wyniki}

Ostatecznym wynikiem jest utworzenie szkieletu aplikacji
składającej się z~trzech elementów ułożonych kaskadowo, które
mogą być uruchomione równolegle celem zwiększenia szybkości wykonywania:

\begin{enumerate}
    \item Detektor frontu.
    \item Detektor numeru we froncie.
    \item Czytnik numeru.
\end{enumerate}

Detektor frontów podczas pisania tego tekstu prezentował skuteczność
na więcej niż zadowalającym poziomie. Do pełnej weryfikacji 
wymagane byłoby przygotowanie środowiska umożliwiającego przeprowadzenie
w~pełni automatycznych testów.

Detektor numeru musiał zostać wspomożony ograniczeniem obszaru wyszukiwania
do lewego górnego rogu odnalezionego frontu autobusu. Niestety 
jest to najsłabszy element i~wymaga usprawnień w~pierwszej
kolejności o~ile nie wymiany na zupełnie inne rozwiązanie.

Detektory poszczególnych cyfr wprowadzone zostały w~celu ograniczenia
złożoności obliczeniowej podczas dopasowywania wzorców. Ich skuteczność
jest daleka od oczekiwanej lecz problemy podczas trenowania detektorów
o~większej ilości próbek (negatywnych i~pozytywnych)
utrudniają znacząco przygotowywanie nowych instancji.

\section{Problemy i~utrudnienia}

Największym utrudnieniem okazało się błędne podejście do zagadnienia. 
O~ile nie do uniknięcia było wykonanie doświadczeń i~eksperymentów,
które zakończyły się jedynie poszerzeniem posiadanej wiedzy.
To błędnym było wykonywanie ogromnej ilości prób i~pomiarów na
wczesnym etapie poszukiwań odpowiedniego rozwiązania. 
Wynikiem takiego podejścia były testy wykonane przy zbyt małej 
wiedzy na temat zagadnienia i~architektury końcowej programu. 

Rzetelne testy wydajności i~dokładności należało wykonać na końcu
gdy wiadomo było już jakie algorytmy, narzędzia i~biblioteki zostaną
zastosowane w~rozwiązaniu docelowym. Niestety z~tego powodu
brak jakichkolwiek statystycznych danych na temat skuteczności
bądź wydajności. Dostępne są jedynie jednostkowe wyniki uzyskane po
przeprowadzeniu manualnych testów programu uruchomionego na telefonie.
Wyniki dostarczone w~ten sposób wykazały, że narzędzie działa,
w~ponad połowie przypadków zwraca poprawny wynik, a~sama aplikacja
wymaga jeszcze ogromnego nakładu pracy. Czy to na usunięcie błędów,
poprawienie skuteczności, wydajności, czy przygotowanie 
wspomnianego już środowiska testowego. Jak wiadomo bez możliwości
przeprowadzenia pomiaru trudno sprawdzić czy coś zostało 
poprawione czy wręcz przeciwnie.

\section{Dalsze prace}

Pierwszym krokiem w~celu przygotowania aplikacji na poziomie 
skuteczności, niezawodności i~użyteczności pozwalającym na 
wydanie komercyjne jest przygotowanie środowiska testowego. 
Swoisty lab miałby służyć nie tylko testom statystycznym ale 
umożliwiać również odtwarzanie błędów zgłaszanych przez użytkowników.

Możliwe są zmiany w~implementacji niosące za sobą większą funkcjonalność.
Może to być na przykład wykrywanie innych typów frontów niż 
tych z~wyświetlaczem. Cel można osiągnąć poprzez zastosowanie 
dodatkowych, specjalnie w~tym celu przygotowanych detektorów. Można też 
podjąć próbę implementacji detektorów tablic bocznych a także tyłów 
autobusów, gdzie oba te fragmenty zawierają numer linii.

Usprawnieniem i~pewnego rodzaju ułatwieniem odczuty byłoby zawężenia
możliwych do odczytania numerów do zbioru z~konkretnego przystanku. 
Współrzędne przystanków wraz z~listą numerów byłyby umieszczane 
w~centralnej bazie. Aplikacja mogła by natomiast pobierać aktualne 
współrzędne z~urządzenia GPS zainstalowanego w~telefonie i~na ich podstawie
ograniczać możliwy do odczytania zbiór do pobranej listy numerów.

Poprawę działania algorytmu można również osiągnąć wykorzystując 
lepszy aparat telefoniczny. Zawierający lepszej jakości kamerę oraz
większą moc obliczeniową. Na rynku dostępne są już urządzenia
wyposażone w~procesory ośmiordzeniowe z~kamerami umożliwiającymi 
nagrywanie filmów w~rozdzielczości 4K.

Niewielką zmianą możliwą do zaimplementowania jest też wprowadzenie
tłumaczeń odczytywanych komunikatów na język użytkownika. 
Jest to zmiana raczej kosmetyczna jednak znacząca jeżeli chodzi
o~aplikacje komercyjne.

Ostatecznie istnieje szereg możliwości wzbogacania i~dodawania
funkcjonalności aplikacji w~oparciu o~użyte technologie.
Można na przykład dopisać moduł lokalizacji przystanków
na podstawie znaków - zadanie o~wiele łatwiejsze. Wymagało by to
wyszkolenia kolejnego detektora. W~tym przypadku, detektora znaków
występujących przy przystankach. Aplikacja przy wykorzystaniu 
modułu GPS była by w~stanie doprowadzić osobę do przystanku.

Można dodać obsługę rozpoznawania mowy, w~celu zadania punktu docelowego.
Przy tak zdefiniowanych kryteriach aplikacja używając komunikatów
głosowych prowadziła by użytkownika do najbliższego przystanku. 
Asystowała by też przy poprawnym ustawieniu telefonu w~celu odczytania
numeru. Podczas podróży możliwe było by aktualizowanie czasu oraz
odległości pozostałej do osiągnięcia celu. O~czym użytkownik informowany
był by przy pomocy osobnych komunikatów głosowych.

Jak widać możliwości rozbudowywania proponowanego narzędzia są niemal
nieskończone. Jedynym kryterium jakie powinno obowiązywać przy
doborze funkcji do zakodowania jest ich przydatność dla użytkowników 
końcowych. To oni tak na prawdę są największymi beneficjentami 
potencjalnego systemu i~bez konsultacji z~przynajmniej jedną osobą
niewidomą nie ma sensu rozpoczynać kolejnej przygody.

Największą
wartością dodaną tego opracowania jest stwierdzenie możliwości odczytania
numeru nadjeżdżającego autobusu przez urządzenie mobilne.
Osiągnięcie poziomu niezawodności pozwalającego osobie niewidomej 
na swobodne poruszanie się i~korzystanie z~transportu miejskiego jest
realne lecz wymaga ogromnego nakładu pracy. Pytaniem nad którym 
trzeba się poważnie zastanowić jest sens istnienia takiego programu.
W~obecnym momencie bowiem, nie istnieją jeszcze autobusy bezzałogowe. 
Bardziej prawdopodobnym jest więc tylko, 
brak innych ludzi na przystanku, z~którymi osoba niewidoma mogłaby się
skonsultować. Problem pojawia się więc tylko gdy potencjalny użytkownik,
znajduje się na przystanku ,,na żądanie'', a~bok
nie ma nikogo, kto mógłby zatrzymać nadjeżdżający autobus. W~takim
przypadku dużym ułatwieniem dla użytkownika byłoby narzędzie
które zasygnalizuje wystąpienie nadjeżdżającego autobusu.
Bez konieczności odczuty numeru linii. Po zatrzymaniu
zawsze można zapytać kierowcy o~tenże numer. Chyba, że on 
z~koli jest\ldots głuchoniemy.
