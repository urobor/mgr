\chapter{Podsumowanie i~wnioski}

Praca ta miała na celu zidentyfikowanie odpowiedniej metody
odczytu numeru nadjeżdżającego autobusu oraz jej implementację
na urządzeniu mobilnym z~systemem Android. 
Sporządzono przegląd algorytmów, narzędzi i~bibliotek
potencjalnie przydatnych podczas rozwiązywania omawianego zagadnienia.
Wykorzystując bibliotekę OpenCV stworzono aplikację 
opartą na rzetelnych i~sprawdzonych rozwiązaniach.
Poprzez serię testów i~wielokrotne dostrajanie oraz konfigurację poszczególnych elementów
programu udało się zbudować narzędzie
posiadające zakładaną funkcjonalność. Wiele wysiłku i~cierpliwości
wymagało przygotowanie nie tylko dużych ale i~wartościowych
zbiorów obrazów służących do uczenia detektorów kaskadowych - głównych
elementów składowych prezentowanego rozwiązania. Wiele czasu 
poświęcono na przygotowanie i~przeprowadzenie procedur
testowych wytworzonego oprogramowania. Zarówno proces uczenia
maszynowego jak i~późniejsze testy przeprowadzono wielokrotnie,
korygując przy tym wykryte błędy w~założeniach, przygotowanych 
zbiorach obrazów oraz identyfikując i~wykorzystując napotkane 
trendy oraz tendencje detektorów do wykrywania określonych cech w~obrazach.

Dzięki modułowej architekturze aplikacji z łatwością można było 
udoskonalać lub zupełnie wymieniać elementy odpowiedzialne za poszczególne 
kroki proponowanej kaskady:
- wyszukanie frontu autobusu w~scenie,
- lokalizacja numeru linii we froncie,
- odczytanie ciągu znaków z~fragmentu obrazu reprezentującego numer linii.
Wielokrotnie poprawiano i~dopracowywano poszczególne 
moduły, wraz z~pojawianiem się nowych danych w~postaci wyników przeprowadzonych testów.
Głównym obiektem modyfikacji były parametry wejściowe narzędzia służącego do przygotowywania
detektorów kaskadowych dostępnego w~ramach pakietu OpenCV. W~końcowym
etapie prac zupełnie przebudowany został ostatni moduł odpowiedzialny za odczytywanie
numeru linii. Zrezygnowano z~intensywnej obliczeniowo metody dopasowywania
wzorca na rzecz detektorów kaskadowych.

Przygotowany prototyp aplikacji nie jest ani optymalny, ani nie 
pokrywa wszystkich przypadków szczególnych. W~sprzyjających warunkach
jest jednak wyjątkowo skuteczny, a~modułowa architektura umożliwia
wprowadzanie usprawnień i~modyfikacji, które pozwolą na obsługę większej
liczby niestandardowych scenariuszy.
W~testach terenowych na kilkadziesiąt dokonanych 
  odczytów odnotowano jedynie dwie pomyłki wprowadzające użytkownika w~błąd.
Kilka prób zakończyło się serią poprawnych odczytów, z~pojedynczymi tylko błędami. 
Proponowane rozwiązanie
jest jednak dalekie od stanu pozwalającego na publikację
i~udostępnienie go szerszemu gronu użytkowników. Specyfika pracy takiego narzędzia,
czyli pomoc osobom niewidomym, wymaga bardzo wysokiej niezawodności.
Komercyjna aplikacja powinna obsługiwać więcej przypadków użycia niż
zostało przewidzianych, opracowanych i~opisanych na łamach niniejszej
pracy. W~normalnych warunkach wszelkie prace powinny zostać
poprzedzone rzetelną analizą potrzeb i~wymagań stawianych projektowanemu 
narzędziu.

\section{Napotkane problemy}

Dwa główne problemy podczas przygotowywania
oprogramowania wykorzystującego proces uczenia 
maszynowego to czasochłonność i~ogromne ilości danych wejściowych.
Wielokrotne poprawki parametrów wejściowych i~samych zbiorów obrazów:
pozytywnych i~negatywnych zajmowały bardzo dużo czasu i~ostatecznie
trwały długo. Dodatkowo sam proces uczenia detektorów trwał od kilku godzin do 
kilku dni. Poprawki były możliwe do naniesienia dopiero po wykonaniu, najkrótszych, bo 
kilkuminutowych testów skuteczności. Wielokrotnie powtarzany proces
uczenia detektorów, wykonywania testów, analizy wyników i~nanoszenia poprawek 
trwał bardzo, ale to na prawdę bardzo długo. 
Wypracowywanie metodologii i~sposobów postępowania 
również zajęło sporo czasu. Ostatecznie, dzięki dopracowaniu procedur
testowych przygotowano detektory cechujące się skutecznością na zadowalającym poziomie.
Co ważne, poziomie (w~większym lub mniejszym stopniu) rzetelnie weryfikowalnym.

Jeżeli chodzi o~problemy związane z~samym zagadnieniem
to można wyodrębnić trzy główne grupy:

\begin{itemize}
	\item zakłócenia utrudniające odczyt,
	\item różnorodność modeli autobusów, sposobów prezentacji numeru, czcionek itp,
	\item niska rozdzielczość współczesnych aparatów.
\end{itemize}

Zakłócenia uniemożliwiające odczyt to miedzy innymi: ludzie stojący
na przystanku zasłaniający czoło autobusu, opady atmosferyczne,
odblaski słoneczne widoczne w~szybie osłaniającej numer uniemożliwiające jego
odczytanie. Jest to dziedzina problemów, które jeżeli wystąpią,
czynią proponowane narzędzie bezużytecznym, nieważne ile pracy
poświęci się na usprawnienia. W~tych sytuacja informacje 
o~numerze nie są zawarte w~pobieranych zdjęciach, przez co niemożliwe
jest ich odczytanie - danych po prostu nie ma.

Zaobserwowanym utrudnieniem było też zjawisko migotania wyświetlacza, przez
co numery były widoczne częściowo lub były zupełnie wygaszone.
Podejście statystyczne, które wykorzystywało kilka obrazów
w~procesie jednej sesji odczytu numeru skutecznie rozwiązało ten problem.

Kolejne zagadnienie było bardziej wyzwaniem niż utrudnieniem. Chodzi
bowiem o~ogromną różnorodność autobusów, sposobów prezentacji numeru,
wykorzystanych krojów czcionek itp. Wiąże się to z~dogłębną analizą
i~przygotowaniem odpowiedniej wersji aplikacji dla danego kraju,
miasta, bądź nawet przystanku.

Ostatnim problemem ograniczającym funkcjonalność omawianej 
aplikacji była rozdzielczość przetwarzanych
obrazów, równa 1920x1440 pikseli.
Była ona ograniczona przez implementację biblioteki OpenCV dostępnej
na platformie Android. Wykorzystane podejście - podgląd klatki 
na wyświetlaczu - wprowadzało ograniczenie w~postaci rozdzielczości
właśnie tego wyświetlacza. Testowe urządzenie - telefon LG G3 - 
umożliwiało nagrywanie obrazu w~rozdzielczości 4K (3840x2160) czyli
prawie dwukrotnie większej. Rozwiązanie tego ograniczenia wymagałoby jednak przygotowania
własnej implementacji obsługującej pobieranie i~przetwarzanie 
obrazów z~kamery. Większa rozdzielczość oznaczałaby większą 
złożoność obliczeniową. Z~drugiej jednak strony umożliwiłoby to
poprawne odczytanie numeru z~większej odległości.

\section{Dalsze prace i~usprawnienia}

Dalsze prace powinny rozpocząć się od zrobienia kroku w~tył.
Podejmując ogromny wysiłek jaki niewątpliwie związany byłby z~doprowadzeniem 
proponowanej aplikacji do stanu użyteczności, trzeba zadać sobie pytanie
czy ma to sens. Niezbędnym jest przeprowadzenie rzetelnej analizy
biznesowej, wywiad z~osobami niewidomymi i~zebranie ich wymagań.
Bez zainteresowania ze~strony docelowej grupy użytkowników końcowych
popełnianie jakiejkolwiek aplikacji nie miałoby najmniejszego sensu.

W~następnej kolejności należałoby się skupić na problemach, które już zidentyfikowano.
Mechanizm odczytujący numer 
linii autobusowej powinien zostać przebudowany i~usprawniony (ostatni element proponowanej kaskady).
Rozsądnym wydaje się przygotowanie w~tym celu mechanizmu opartego na sieciach
neuronowych, które to ze względu na złożoność konfiguracji i~niekiedy
dłuższe czasy przygotowywania niż opisane detektory kaskadowy zostały
pominięte w~sposób zupełnie świadomy.

Można wymieniać wiele możliwych usprawnień:
\begin{itemize}
	\item przygotowanie zbioru oczekiwanych numerów na podstawie współrzędnych
	GPS i~danych z~bazy przystanków i~przypisanych do nich możliwych numerów linii,
	\item przygotowywanie detektorów dla poszczególnych modeli autobusów,
	\item wymiana lub usprawnienia posiadanych elementów systemu,
	\item wprowadzanie dodatkowego kroku weryfikującego czy wykryty 
	fragment rzeczywiście jest szukanym obiektem,
	\item wykorzystanie telefonu z~obiektywem o~zmiennej ogniskowej,
	umożliwiającym wykonanie optycznego powiększenia obrazu,
	\item itd.
\end{itemize}

Po głębszym zastanowieniu, drugim krokiem powinien być następny
etap analizy. Należałoby ustalić pierwotny obszar wykorzystania 
hipotetycznej aplikacji. Czy odczytywane mają być numery autobusów,
czy może też tramwajów. Czy aplikacja ma być wykorzystywana na terenie
konkretnego kraju, wielu krajów, w~konkretnym mieście, czy może ma mieć zasięg
globalny. Nie istnieje bowiem ogólny i~uniwersalny wzorzec frontu pojazdu
odnośnie którego można by przygotować równie uniwersalny i~skuteczny
zestaw detektorów - nieistotne jakiego typu (co prawda wykorzystanie
sieci neuronowych może nieść ze sobą spektakularne rezultaty, jednak
wymaga to ogromnych wręcz zbiorów obrazów potrzebnych do ich wytrenowania).
Odpowiedzi na tak postawione pytania mogą mieć wpływ
na architekturę rozwiązania. Jeżeli aplikacja swym zasięgiem 
miałaby objąć cały świat to z~pewnością powinna być bardziej konfigurowalna, 
łatwiej rozszerzalna o~dodatkowe moduły dodające nowe funkcjonalności.
Aplikacja dostępna w~obecnej formie, najprawdopodobniej poradzi sobie
w~ponad osiemdziesięciu procentach przypadków wykorzystania w~Warszawie, i~być
może w~innych dużych miastach w~naszym kraju. Zaprojektowanie i~napisanie
aplikacji uniwersalnej, konfigurowalnej i~niezawodnej, z~perspektywy
doświadczeń nabytych podczas pisania tej pracy, jest jak najbardziej
możliwe, jednak stanowczo wykracza poza jej zakres.

Biorąc pod uwagę doświadczenia zawodowe, niemal zawsze próba napisania
uniwersalnego i~konfigurowalnego narzędzia kończy się koszmarem podczas utrzymania i~rozwijania 
ogromnej bazy skomplikowanego kodu. Niejednokrotnie próby rozszerzenia bądź konfiguracji
kończą się na pisaniu dziwacznego zestawu trików w~kodzie, którego autor akurat tego przypadku
nie przewidział, a~sama aplikacja nie jest tak konfigurowalna jakby się tego chciało.
Więc nowe pomysły i~usprawnienia owszem są możliwe do wykonania jednak rozsądnym byłoby
powrócenie i~skupienie się na etapie analizy zagadnienia.

Niezaprzeczalną wartością dodaną pozyskaną podczas pisania tej pracy
jest zdobyta wiedza z~dziedziny uczenia maszynowego. Jak ważny jest
dobrze przygotowany zbiór obrazów pozytywnych i~negatywnych. Ile znaczy
skuteczna procedura testowa i~jakie znaczenie ma liczebność zbioru danych
na których testowane są wyszkolone detektory. Jest to wiedza na tyle 
uniwersalna, że z~pewnością nie ogranicza się jedynie do detektorów
kaskadowych Violi-Jonesa. Podobne mechanizmy występują we wszystkich
algorytmach i~narzędziach wykorzystujących uczenia maszynowe, w~tym modnych
ostatnio sieciach neuronowych. Dzięki zdobytym doświadczeniom, każda
następna próba rozwiązania jakiegokolwiek zagadnienia z~wykorzystaniem uczenia maszynowego
będzie z~pewnością bardziej przemyślana, świadoma i~co za tym idzie o~wiele skuteczniejsza.

