% streszczenie
\begin{center}
	\textbf{Streszczenie}
\end{center}

Celem pracy była implementacja aplikacji możliwej do uruchomienia
na urządzeniu mobilnym z~systemem android, której zadaniem
byłoby odczytywanie numeru linii z~frontu nadjeżdzającego autobusu.
W~pierwszej częśći pracy zamieszczono przegląd narzędzi, algorytmów,
bibliotek i~frameworków służących do wykrywania obiektów w~scenach
oraz ich identyfikacji - rozpoznawania. 
Następnie zamieszczono opis przeprowadzonych eksperymentów
mających na celu wyłonienie najpewniejszych rozwiązań, które
zostałyby wykorzystane do skonstruowania ostatecznej implementacji.
W~pracy zawarty został też opis prób podjętych w~celu automatyzacji
procesu przygotowywania oraz weryfikacji powstających narzędzi 
i~programów.
Podsumowując, niniejsze opracowanie jest 
swoistym dziennikiem opisującym przebieg procesu myślowego
od studium dostępnych rozwiązań, poprzez szereg przygotowań
i~eksperymentów, aż do implementacji działającej aplikacji
prototypowej na telefonie z~Androidem.

\vspace*{\baselineskip}

\noindent\textbf{Słowa kluczowe:} \textit{Android, OpenCV, Tesseract,
wizja komputerowa, detektor, wykrywanie obiektów, rozpoznawanie obiektów,
OCR, autobus, numer autobusu, odczytywanie numeru autobusu.}

\vspace*{2\baselineskip}

\begin{center}
	\textbf{Abstract}
\end{center}

IN ENGLISH\ldots 

\vspace*{\baselineskip}

\noindent\textbf{Keywords:} \textit{keyword1, keyword2.}

\setcounter{page}{2}
