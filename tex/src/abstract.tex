% streszczenie
\begin{center}
	\textbf{Streszczenie}
\end{center}

W~ramach pracy wykonano projekt oraz implementację aplikacji możliwej do zainstalowania 
i~uruchomienia
na urządzeniu mobilnym z~systemem Android. Głównym i~jedynym 
zadaniem wytworzonego oprogramowania
było odczytywanie numeru linii z~frontu nadjeżdżającego autobusu.
Zgodnie z~architekturą, program po uruchomieniu na telefonie
wchodził w~tryb pobierania klatek obrazu z~aparatu urządzenia. 
Po pobraniu zadanej z~góry liczby klatek z~wystąpieniem frontu
autobusu następował odczyt.

W~pierwszej części pracy zamieszczono przegląd narzędzi, algorytmów,
bibliotek i~struktur ramowych służących do wykrywania obiektów w~scenach
oraz ich identyfikacji - rozpoznawania. Przygotowany w~ten sposób
zbiór narzędzi i~technik wykorzystano w~celu wyłonienia tych najbardziej
adekwatnych w~danym zagadnieniu. Główne kryteria przy doborze to m.in:
skuteczność, łatwość implementacji, dostępna dokumentacja oraz możliwość
konfiguracji.

Następnie zamieszczono opis przeprowadzonych eksperymentów
mających na celu wyłonienie najpewniejszych rozwiązań, które
zostały wykorzystane do skonstruowania ostatecznej implementacji.
W~pracy zawarty został też opis prób podjętych w~celu automatyzacji
procesu przygotowywania oraz weryfikacji powstających narzędzi 
i~programów.

Praca dokumentuje i~systematyzuje zastosowane pomysły i~metody.
Część okazała się nietrafiona, co skutecznie uniemożliwiło ich
wykorzystanie w~rozwiązaniu docelowym. W~wyniku
bardzo licznych eksperymentów udało się wyłonić zbiór 
narzędzi, dzięki którym implementacja była wykonalna.

Wykonane testy skuteczności pomogły w~określeniu niezawodności,
a~przez to przydatności wykonanej aplikacji. Na szczęście
modułowa budowa programu umożliwiała doskonalenie oraz
wymianę poszczególnych jej elementów. Ścieżki rozwoju
i~możliwości usprawnień zostały przedstawione w~ostatniej
części pracy.


\vspace*{\baselineskip}

\noindent\textbf{Słowa kluczowe:} \textit{Android, OpenCV, Tesseract,
wizja komputerowa, detektor, wykrywanie obiektów, rozpoznawanie obiektów,
OCR, autobus, numer autobusu, odczytywanie numeru autobusu.}

\vspace*{2\baselineskip}

\begin{center}
	\textbf{Abstract}
\end{center}

IN ENGLISH\ldots 

\vspace*{\baselineskip}

\noindent\textbf{Keywords:} \textit{keyword1, keyword2.}

\setcounter{page}{2}
