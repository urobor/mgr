% streszczenie
\begin{center}
	\textbf{Streszczenie}
\end{center}

W~ramach pracy wykonano projekt oraz implementację aplikacji możliwej do zainstalowania 
i~uruchomienia
na urządzeniu mobilnym z~systemem Android. Głównym i~jedynym 
zadaniem wytworzonego oprogramowania
było odczytywanie numeru linii z~frontu nadjeżdżającego autobusu.
Zgodnie z~architekturą, program po uruchomieniu na telefonie
wchodził w~tryb pobierania klatek obrazu z~aparatu urządzenia. 
Po pobraniu zadanej z~góry liczby klatek z~wystąpieniem frontu
autobusu następował odczyt.

W~pierwszej części pracy zamieszczono przegląd narzędzi, algorytmów,
bibliotek i~struktur ramowych służących do wykrywania obiektów w~scenach
oraz ich identyfikacji - rozpoznawania. Przygotowany w~ten sposób
zbiór narzędzi i~technik wykorzystano w~celu wyłonienia tych najbardziej
adekwatnych w~danym zagadnieniu. Główne kryteria przy doborze to m.in:
skuteczność, łatwość implementacji, dostępna dokumentacja oraz możliwość
konfiguracji.

Następnie zamieszczono opis przeprowadzonych eksperymentów
mających na celu wyłonienie najpewniejszych rozwiązań, które
zostały wykorzystane do skonstruowania ostatecznej implementacji.
W~pracy zawarty został też opis prób podjętych w~celu automatyzacji
procesu przygotowywania oraz weryfikacji powstających narzędzi 
i~programów.

Praca dokumentuje i~systematyzuje zastosowane pomysły i~metody.
Część okazała się nietrafiona, co skutecznie uniemożliwiło ich
wykorzystanie w~rozwiązaniu docelowym. W~wyniku
bardzo licznych eksperymentów udało się wyłonić zbiór 
narzędzi, dzięki którym implementacja była wykonalna.

Wykonane testy skuteczności pomogły w~określeniu niezawodności,
a~przez to przydatności wykonanej aplikacji. Na szczęście
modułowa budowa programu umożliwiała doskonalenie oraz
wymianę poszczególnych jej elementów. Ścieżki rozwoju
i~możliwości usprawnień zostały przedstawione w~ostatniej
części pracy.


\vspace*{\baselineskip}

\noindent\textbf{Słowa kluczowe:} \textit{Android, OpenCV, Tesseract,
widzenie komputerowe, detektory kaskadowe, 
wykrywanie obiektów, rozpoznawanie obiektów,
OCR, odczytywanie numeru autobusu.}

\vspace*{2\baselineskip}
\newpage

\begin{center}
	\textbf{Abstract}
\end{center}

This paper describes process of design and implementation
of an Android application which could help or even replace 
humans at reading city bus line numbers from the front 
part of the aproaching vihicle. Initial design assumed
that first prototype application would have very limited functionality.
Just after first lunch, program would try to find bus front 
in each frame taken from device camera. After first positive
check, fixed number of frame shall be taken and processed
in case of bus front identification and localization, number
localization and filally recognition of digits in the identified number.

First described activity was the review of available tools, algorythms,
libraries and frameworks which could be helpful in object
detection and recognition in natural scenes. Preapred set of
tools and techniques was reviewed to choose most usefull ones
for each problem. Main criteria were: efficiency and effectiveness,
ease of implementation,
available documentation and possible configuraion.

In next chapter description of conducted experiments was made.
Performed tests were supposed to help in finding 
best suited algorithms and implementation of those
algorithms according to the problem which shall be solved.
After many hours and days spent on preparing and
executing tests the fixed set of tools was chosen 
to make final implementation. At this stage it was 
certain that application was doable.

Extensive test results allowed to determine the effectiveness
and hence the usefulness of proposed applicaiton.
Focus put on modularity led to the fact that implemented
software was easly extendable. Every step of algorithm 
could be improved separately or even raplaced whith completely
different one. Paths of development and improvement
were described in the last chapter.

\vspace*{\baselineskip}
\noindent\textbf{Keywords:} \textit{Android, OpenCV, Tesseract,
	cumputer vision, cascade detectors, 
	object detection, object recognition,
	OCR, city bus number reading.}

\setcounter{page}{2}
